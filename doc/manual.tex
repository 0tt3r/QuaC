\documentclass{article}
\usepackage{graphicx}
\usepackage{amsmath}
\usepackage{xcolor}
\usepackage[T1]{fontenc}
\usepackage{listings}
\newenvironment{ttenv}{\begin{center}\ttfamily}{\end{center}}

\begin{document}

\definecolor{mygreen}{rgb}{0,0.6,0}
\definecolor{mygray}{rgb}{0.5,0.5,0.5}
\definecolor{mymauve}{rgb}{0.58,0,0.82}

\lstset{ %
  belowcaptionskip=1\baselineskip,
  breaklines=true,
  frame=L,
  xleftmargin=\parindent,
  language=C,
  showstringspaces=false,
  basicstyle=\ttfamily,
  keywordstyle=\bfseries\color{green!40!black},
  commentstyle=\color{red},
  stringstyle=\color{orange},
}
\title{QuaC User Manual}
\author{Matthew Otten}

\maketitle

\begin{abstract}
  Basic notes on using QuaC, a Lindblad master equation solver written in C.
\end{abstract}

\section{Installation}

\section{Creating a New System}
QuaC was designed to allow the user to go from the physics to the computations with
as little pain as possible. To illustrate this concept, we will use a simple
Jaynes-Cummings Hamiltonian coupled to a thermal resonator
as our physics and show the steps necessary to
program the physics and run the code.

\subsection{Hamiltonian}
The Jaynes-Cummings model describeds the dynamics of
a two level system coupled to an oscillator mode (be it a mechanical resonator, a
plasmonic system, or an electromagnetic field):
\begin{equation}\label{jc_ham}
  H =  \omega_a a^\dagger a +  \omega_\sigma \sigma^\dagger \sigma +
   g (\sigma^\dagger a + \sigma a^\dagger),
\end{equation}
where $a^\dagger$ is the creation operator of the oscillator, $\omega_s$ is the frequency
of the oscillator, $\sigma^\dagger$ is the creation operator of the two level system,
$\omega_\sigma$ is the transition frequency of the two level sistem, and $g$ is the
coupling strength between the two systems. We are using units such that $\hbar = 1$.
To create this system within QuaC, we first
have to create operators for each subsystem, via
\begin{lstlisting}
  create_op(number_of_levels,&op_name)
\end{lstlisting}
where \texttt{number\_of\_levels} is the number of levels of the operator and \texttt{op\_name}
is a previously declared variable of the \texttt{operator} type. The \texttt{operator} type
is a special QuaC object that is used in the creation of the Hamiltonian and Lindblad terms.
One call to \texttt{create\_op}, with a single \texttt{op\_name} creates the annihilation,
creation, and number (i.e. $a^\dagger a$) operators.
\texttt{op\_name} is the annihilation operator within the subsystem's Hilbert space,
\texttt{op\_name->dag} in the creation operator, and \texttt{op\_name->n} is the number operator.
Each of these three should be thought of as opaque objects; internally, they store
all the necessary information
for the creation of the Hamiltonian matrix which uses those terms,
but they should never be directly manipulated by the
user.

For our example, we need to create two operators: one for the oscillator, and one
for the two level system. Trivially, we know the number of levels for the two level system, but
it is not so clear for the oscillator. Formally, the oscillator has infinite levels. Practically,
there will be some maximum excitation number for a given problem. The infinite ladder
should be truncated at a point in which adding more levels has minimal impact on the
dynamics. In this example, we will use 25. Two create our two operators, we call
\begin{lstlisting}
  create_op(2,&sigma)
  create_op(25,&a)
\end{lstlisting}
where \texttt{sigma} and \texttt{a} are previously declared variables of the \texttt{operator} type.
With our operators defined, we can now construct the Hamiltonian matrix. In QuaC, Hamiltonians
are created term by term, with calls to the \texttt{add\_to\_ham} family of functions. To add
a term which has a single operator (including the number operator, $a^\dagger a$, as a
'single operator'), we use
\begin{lstlisting}
  add_to_ham(scalar,op_name)
\end{lstlisting}
where \texttt{scalar} is the scalar that multiplies the operator, \texttt{op\_name}. In our
Jaynes-Cummings model, we need to add two such terms to the Hamiltonian,
$\omega_a a^\dagger a$ and $\omega_\sigma \sigma^\dagger \sigma$. We would call
\begin{lstlisting}
  add_to_ham(omega_a,a->n)
  add_to_ham(omega_s,sigma->n)
\end{lstlisting}
where \texttt{omega\_a} and \texttt{omega\_s} are previously declared (and assigned) numbers,
and \texttt{a->n} and \texttt{sigma->n} are the number operators for the oscillator
and two level system.
The coupling terms in our Hamiltonian have two operators
($g (\sigma^\dagger a + \sigma a^\dagger)$), so we will need to use
\begin{lstlisting}
  add_to_ham_mult2(scalar,op_name1,op_name2).
\end{lstlisting}
Since we have the sum of two such terms, we will need to call this function twice:
\begin{lstlisting}
  add_to_ham_mult2(g,sigma->dag,a)
  add_to_ham_mult2(g,sigma,a->dag)
\end{lstlisting}
where $g$ is a previously declare and assigned number. Here we have used both the
annihilation and creation operators of the \texttt{operator} type. We have now taken
our initial Hamiltonian of eq.~(\ref{jc_ham}) and told QuaC everything it needs
to do a calculation using that Hamiltonian. 
\subsection{Lindblad Terms}
QuaC is optimized for open quantum systems. As such, we need to 
include nonunitary evolution. QuaC does this
through the Lindblad superoperator. The Lindblad superoperator is defined as
\begin{equation}
  L(C)\rho = C\rho C^\dagger - \frac{1}{2}(C^\dagger C \rho + \rho C^\dagger C),
\end{equation}
where $C$ is an operator. The Lindblad superoperator can effectively describe
dissipation and dephasing. For example, our TLS might have some spontaneous emission
with rate $\gamma_\sigma$. This could be included using a Lindblad term of
the form $\gamma_\sigma L(\sigma)\rho$ in our master equation. In QuaC, the general
function for adding Lindblad terms is
\begin{lstlisting}
  add_lin(scalar,op_name)
\end{lstlisting}
where \texttt{scalar} is the rate, and \texttt{op\_name} is the associated operator. In our example,
we want to add $\gamma_\sigma L(\sigma)\rho$, so we call
\begin{lstlisting}
  add_lin(gamma_s,sigma)
\end{lstlisting}
where \texttt{gamma\_s} is the (previously declared and assigned) spontaneous emission rate
of our TLS and \texttt{sigma} is the annihilation operator which we previously created.
Our oscillator can also have Lindblad terms. For this example, let's say that the oscillator
is coupled to a thermal bath with thermal occupation number $n_{th}$ and has a linewidth
$\gamma_a$. This would give two Lindblad terms:
$\gamma_s (n_{th} + 1) L(a)\rho + \gamma_s n_{th} L(a^\dagger)\rho$. To add these terms in
QuaC, we call
\begin{lstlisting}
  add_lin(gamma_a*(n_th+1),a)
  add_lin(gamma_a*n_th,a->dag)
\end{lstlisting}
We could, of course, add additional Lindblad terms (such as environmental dephasing), but,
for our simple example, this will suffice.
\subsection{Full QuaC Code}
  For illustrative purposes, we collect
all the calls needed here:
\begin{lstlisting}
  create_op(2,&sigma) //Create the TLS
  create_op(25,&a)    //Create the oscillator

  //Add terms to the Hamiltonian
  add_to_ham(omega_a,a->n)      
  add_to_ham(omega_s,sigma->n)
 
  add_to_ham_mult2(g,sigma->dag,a)
  add_to_ham_mult2(g,sigma,a->dag)

  //Add Lindblad terms
  add_lin(gamma_s,sigma)
  add_lin(gamma_a*(n_th+1),a)
  add_lin(gamma_a*n_th,a->dag)
\end{lstlisting}
Neglecting variable declaration and assignment, these nine lines of code are all that is necessary
to construct our simple Jaynes-Cummings system. In the next section, we will discuss
how to use the constructed system to solve for both the steady state and time dependence.

\section{Solving the System}
QuaC has two modes for using the constructed system: solving for the steady state of the system,
and solving for the full time dependence.
\subsection{Steady State}
Solving for the steady state in QuaC simply involves a call to
\begin{lstlisting}
  steady_state()
\end{lstlisting}
Once this call is executed, QuaC solves for the steady state of the system, using, as default,
additive Schwarz method preconditioned GMRES. The selection of algorithm can be changed
at runtime using PETSc's flexible command line interface (see section on utilizing
PETSc). The initial condition is unimportant, as the solver goes to the
(unique?) steady state from any initial condition rapidly (maybe?).
TODO: Add information about getting population/concurrence/etc after the solve and results

\subsection{Time Dependence}
To solve for the time dependence, we must first set the initial conditions. Initial conditions
are set per subsystem, and the system is restricted to starting in a pure state (though
this limitation will be removed in a future version). If no initial condition is set
for a subsystem, it is assumed to be in the $| 0\rangle$ state initially. In QuaC, to set
the initial population of a subsystem, you call
\begin{lstlisting}
  set_initial_pop(op_name,pop)
\end{lstlisting}
where \texttt{op\_name} is an operator and \texttt{pop} is the population to add.
\texttt{op\_name} can be any of the three basic operators created by QuaC in the initial
\texttt{create\_op} call; there is no difference between the three for this routine. By convention,
it is best to use the annihilation operator. \texttt{pop} will be understood as an
integer, and must be less than the total number of levels for that subsystem. This stems
from the fact that \texttt{set\_initial\_pop} is really setting the initial condition of the
subsystem as a pure state of the form $|\texttt{pop}\rangle\langle\texttt{pop}|$. For our
Jaynes-Cummings example, we would call
\begin{lstlisting}
  set_initial_pop(a,pop_a)
  set_initial_pop(sigma,pop_sigma)
\end{lstlisting}
With the initial conditions set, we can now proceed to do the timestepping using
\begin{lstlisting}
  time_step(time_max,dt,steps_max)
\end{lstlisting}
where \texttt{time\_max} is the maximum time to integrate to, \texttt{dt} is the initial
time step, and \texttt{steps\_max} is the maximum number of time steps to take. The default
solver is an explicit, adaptive time step Runge-Kutta method. QuaC (through PETSc) also
supports other explicit, as well as implicit, time steppers. Algorithms can be chosen
at runtime; see the section on PETSc to learn more about these options. QuaC will
then run the time dynamics until it either reaches the end time or completes
the maximum number of steps.

\subsubsection{Printing Results at Each Time Step}
To get results at each time step, a user defined function must be declared and passed to
QuaC via the function
\begin{lstlisting}
  set_ts_monitor(function_name)
\end{lstlisting}
\texttt{function\_name} must have this particular form:
\begin{lstlisting}
PetscErrorCode function_name(TS ts,PetscInt step,PetscReal time,Vec dm,void *ctx)
\end{lstlisting}
where \texttt{ts} is an opaque object which holds information about the timestepping algorithm
(and need not be touched by the user), \texttt{step} is the current step number,
\texttt{time} is the current time, \texttt{dm} is the (vectorized) density matrix (and should
be though of as an opaque object to be passed to other routines), and \texttt{ctx} is another
opaque object that the user should not touch. Within the ts\_monitor function,
the user can
call a variety of functions to obtain and print different observables and metrics.
To print the current populations to a file name \texttt{pop}, you call
\begin{lstlisting}
  get_populations(dm,time)
\end{lstlisting}
This routine will calculate the populations for each subsystem and print them to the file
\texttt{pop}. TODO: Add observables (\texttt{get\_observables}), concurrence, fidelity, etc
The ts\_monitor function must have \texttt{PetscFunctionReturn(0)} as its final line.

\section{Unequally Spaced Operators}

\section{Using PETSc Command Line Options}

\end{document}
